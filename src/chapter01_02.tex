
\section{Matrices}

Mientras no se indique lo contrario, las matrices son cuadradas, de dimensión $n$ y sobre el cuerpo de los números complejos.

Si $\mathbf{A}$ es una matriz, llamamos a sus elementos $\mathbf{A}_i^j$. A la $i$-ésima fila de $\mathbf{A}$ la denotamos por $\mathbf{A}_i$ y a la $i$-ésima columna de $\mathbf{A}$ la denotamos por $\mathbf{A}^i$.

Dada una matriz $\mathbf{A}=(\mathbf{A}_i^j)$ podemos definir los siguientes conceptos:
\begin{tabular}{l|l|l}
	Mombre & Notación & Definición \\
	\hline
	Matriz adjunta & $\mathbf{A}^T$ & ${\mathbf{A}^T}_i^j := \mathbf{A}_j^i$. \\
	Diagonal & $\text{diag}(\mathbf{A})$ & $(\mathbf{A}_1^1, \dots, \mathbf{A}_n^n)$. \\
	& $\mathbf{A}|_i^j$ & $\mathbf{A}$ quitando la $i$-ésima fila y la $j$-ésima columna. \\
	menor de $\mathbf{A}_i^j$ & & det$(\mathbf{A}|_i^j)$. \\
	cofactor de $\mathbf{A}_i^j$ & $\mathbf{\tilde{A}}_i^j$ & $(-1)^{i+j}$ det$(\mathbf{A}|_i^j)$. \\
	adjunto & adj$(\mathbf{A})$ & $(\text{adj}(\mathbf{A}))_i^j := \mathbf{\tilde{A}}_i^j$. \\
	$\mathbf{A}$ es ortogonal & & Si $\mathbf{A}\mathbf{A}^T=\mathbf{A}^T\mathbf{A}=\mathbf{I}$.
\end{tabular}

\paragraph{Resultados}
\begin{itemize}
	\item Sea $\mathbf{A}$ una matriz, $\mathbf{A}\text{adj}(\mathbf{A}) = \text{adj}(\mathbf{A})\mathbf{A}=\text{det}(\mathbf{A})\mathbf{I}$.
	\item Sea $\mathbf{A}$ una matriz ortogonal, entonces det$(\mathbf{A})=\pm1$.
\end{itemize}
