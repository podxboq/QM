\section{Espacios Cuánticos}
A partir de este punto, vamos a usar la notación cuántica de Dirac:
\begin{itemize}
  \item Denotaremos por $\mathbb{H}$ a los espacios de Hilbert y a sus vectores por $\ket{x}$.
  \item Llamamos operadores a los homomorfismos entre espacios de Hilbert y los denotamos por $\hat{f}$.
  \item Denotaremos a la función dual de $\ket{x}$ por $\bra{x}$.
  \item Denotaremos al producto escalar de $\ket{x}$ y $\ket{y}$ por $\braket{x}{y}$.
  \item Denotaremos la expresión $\operatoravg{x}{\hat{f}}{y}$ al producto escalar de ${\ket{x}}$ y $\ket{\hat{f}(y)}$.
\end{itemize}

\subsection{Espacios de Hilbert}

Sea $B$ un conjunto, llamamos \define{$L^2(B)$} al conjunto de las aplicaciones $f:B\rightarrow\C$ que cumplen $\sum_{b\in B}\nor{f(b)}^2 < \infty$ junto con el producto interno $\braket{f}{g}=\sum_{b\in B}\overline{f(b)}g(b)$.

\begin{resultado}
  Tanto $L^2(\R)$ como $L^2(\C)$ son espacios de Hilbert.
\end{resultado}
