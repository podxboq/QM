\section{Estadística y probabilidad}
\subsection{Estadística}
Si $L=\{x_i\}$ es una lista finita de $n$ valores, definimos la relación de equivalencia sobre los índices siguiente: $i\sim j$ si $x_i = x_j$. Llamamos $I$ al cojunto cociente, $f(i)$ al cardinal de $[i]$, es decir, el número de veces que el valor $x_i$ se repite en $L$. De esta manera obtenemos el conjunto de pares $\{(x_i, f(i))\ \forall i\in I\}$ y construimos el conjunto formado por dichos pares, y donde  definimos:

\begin{tabular}{l|l|l}
	Mombre & Notación & Definición \\
	\hline
	moda & & El valor más repetido \\
	mediana & & El valor que está en medio de los valores \\
	media & $\overline{L}$ & $\sum_{i=1,\dots,n} x_i/n=\sum_{i\in I}x_if(i)/n$. \\
	desviación & $D_i$ & $x_i-\overline{L}$ \\
	desviación media & $\overline{D}$ & $\sum\abs{D_i}/n$ \\
	varianza & $\sigma^2$ & $\sum D_i^2/n$ \\
	desviación típica & $\sigma$ & $\sqrt{\sigma^2}$\\
\end{tabular}

\subsection{Probabilidad}
llamamos \textbf{frecuencia} del valor $x_i$, al número de veces que se tiene el valor $x_i$ en la lista, es decir $f(x_i)$.

Si $P$ es un suceso aleatorio cuyos resultados están en dicha lista, entonces la \textbf{probabilidad} de obtener el valor $x_i$ es $P(x_i)=\frac{f(x_i)}{n}$, y por tanto, $\sum P(x_i) = 1$, llamamos \textbf{esperanza} a $\avg{L} = \sum x_iP(x_i)$. Es fácil demostrar que $\sigma^2 = \avg{L^2}-\avg{L}^2$.

Para el estudio de procesos probabilísticos continuos, llamamos \textbf{función de distribucción} a $f(x) = P(X\leq x)$, por tanto $\int_{-\infty}^{+\infty}f(x)dx=1$.
