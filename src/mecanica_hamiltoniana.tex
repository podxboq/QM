\chapter{Mecánica Hamiltoniana}

Ahora vamos a considerar el punto de vista de la mecánica clásica a través de la aproximación Hamiltoniana.
El hamiltoniano es considerar la energía del sistema como variables de la posición y del momento.

\begin{definition}[Hamiltoniano]
    Se define el \define{hamiltoniano}{Hamiltoniano} de un sistema como
    \begin{equation}
        \label{hamiltoniano}
        H(x,p)=T+V=\frac{1}{2}\sum_{i=1}^{n} p_i^2 +V(x)
    \end{equation}

\end{definition}

Bajo esta expresión, obtenemos las ecuaciones de Halmiton
\begin{postulate}[Ecuaciones de Halmiton]
    \begin{subequations}
        \begin{equation}\label{eq:halmiton_parcial_momento}
            \frac{\partial H}{\partial p_i}=\frac{d x_i}{dt}
        \end{equation}
        \begin{equation}\label{eq:halmiton_parcial_posicion}
            \frac{\partial H}{\partial x_i}=-\frac{d p_i}{dt}
        \end{equation}
    \end{subequations}
\end{postulate}
