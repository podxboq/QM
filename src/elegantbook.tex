\chapter{ElegantBook Settings}

This template is based on the Standard \LaTeX{} book class, so the options of book class work as well (Note that the option of papersize has no effect due to \lstinline{device} option). The default encoding is UTF-8 while \TeX{} Live is recommended. The test environment is Win10 + \TeX{} Live 2021, either \hologo{pdfLaTeX} or \hologo{XeLaTeX} works fine. \hologo{XeLaTeX} is preferred for Chinese articles.


\section{Color Themes}
This template contains 5 color themes, i.e. \textcolor{structure1}{\lstinline{green}}\footnote{Original default theme.}, \textcolor{structure2}{\lstinline{cyan}}, \textcolor{structure3}{\lstinline{blue}}(default), \textcolor{structure4}{\lstinline{gray}}, \textcolor{structure5}{\lstinline{black}}. You can choose \lstinline{green} with
\begin{lstlisting}
\documentclass[green]{elegantbook} %or
\documentclass[color=green]{elegantbook}
\end{lstlisting}


\begin{table}[htbp]
    \caption{ElegantBook Themes\label{tab:color thm}}
    \centering
    \begin{tabular}{ccccccc}
        \toprule
        & \textcolor{structure1}{green}
        & \textcolor{structure2}{cyan}
        & \textcolor{structure3}{blue}
        & \textcolor{structure4}{gray}
        & \textcolor{structure5}{black}
        & Main Environments\\
        \midrule
        structure & \makecell{{\color{structure1}\rule{1cm}{1cm}}}
        & \makecell{{\color{structure2}\rule{1cm}{1cm}}}
        & \makecell{{\color{structure3}\rule{1cm}{1cm}}}
        & \makecell{{\color{structure4}\rule{1cm}{1cm}}}
        & \makecell{{\color{structure5}\rule{1cm}{1cm}}}
        & chapter section subsection \\
        main & \makecell{{\color{main1}\rule{1cm}{1cm}}}
        & \makecell{{\color{main2}\rule{1cm}{1cm}}}
        & \makecell{{\color{main3}\rule{1cm}{1cm}}}
        & \makecell{{\color{main4}\rule{1cm}{1cm}}}
        & \makecell{{\color{main5}\rule{1cm}{1cm}}}
        & definition exercise problem  \\
        second & \makecell{{\color{second1}\rule{1cm}{1cm}}}
        & \makecell{{\color{second2}\rule{1cm}{1cm}}}
        & \makecell{{\color{second3}\rule{1cm}{1cm}}}
        & \makecell{{\color{second4}\rule{1cm}{1cm}}}
        & \makecell{{\color{second5}\rule{1cm}{1cm}}}
        & theorem lemma corollary\\
        third & \makecell{{\color{third1}\rule{1cm}{1cm}}}
        & \makecell{{\color{third2}\rule{1cm}{1cm}}}
        & \makecell{{\color{third3}\rule{1cm}{1cm}}}
        & \makecell{{\color{third4}\rule{1cm}{1cm}}}
        & \makecell{{\color{third5}\rule{1cm}{1cm}}}
        & proposition\\
        \bottomrule
    \end{tabular}
\end{table}

If you want to customize the colors, please select \lstinline{nocolor} or use \lstinline{color=none} and declare the main, second, and third colors in the preamble section as follows:
\begin{lstlisting}[frame=single]
\definecolor{structurecolor}{RGB}{60,113,183}
\definecolor{main}{RGB}{0,166,82}%
\definecolor{second}{RGB}{255,134,24}%
\definecolor{third}{RGB}{0,174,247}%
\end{lstlisting}


\section{Cover}

\subsection{Customized Cover}
Customized cover is allowed, you can choose or hide any element as you prefer. Current optional elements are:
\begin{itemize}
    \item title: \lstinline{\title}
    \item subtitle: \lstinline{\subtitle}
    \item author: \lstinline{\author}
    \item institute: \lstinline{\institute}
    \item date: \lstinline{\date}
    \item version: \lstinline{\version}
    \item extra information: \lstinline{\extrainfo}
    \item cover image: \lstinline{\cover}
    \item logo: \lstinline{\logo}
\end{itemize}

Besides, an extra command \lstinline{\bioinfo} is provided with two options--caption and content. For instance, if you want to display \lstinline{Username: 111520}, just type in

\begin{lstlisting}
\bioinfo{Username}{115520}
\end{lstlisting}

You can change the color of the horizontal bar of the cover by
\begin{lstlisting}
  \definecolor{customcolor}{RGB}{32,178,170}
  \colorlet{coverlinecolor}{customcolor}
\end{lstlisting}

\subsection{Cover Image}
The cover image size is $1280 \times 1024$.

\subsection{Logo}
Aspect ratio of the logo is 1:1, i.e. a square picture.

\subsection{Stylized Cover}
Want to use stylized cover?. Please comment out \lstinline{\maketitle} and use \lstinline{pdfpages} to insert the
cover. Similar for using \lstinline{titlepage}. If you would like to use the cover in version 2.x, please refer to \href{https://github.com/EthanDeng/etitlepage}{etitlepage}.


\section{Chapter Title Display Styles}

This template contains 2 sets of \textit{title display styles}, \lstinline{hang}(default) and \lstinline{display}
style. For the former, chapter title is displayed on a single line (\lstinline{hang}). For the latter, chapter title is displayed on a double line (\lstinline{display}).In this guide, we use \lstinline{hang} . To change display style, use:
\begin{lstlisting}
\documentclass[hang]{elegantbook} %or
\documentclass[titlestyle=hang]{elegantbook}
\end{lstlisting}


\section{Introduction of Math Environments}
We defined two sets of theorem modes, \lstinline{simple} style and \lstinline{fancy} style (default). You may change to \lstinline{simple} mode by

\begin{lstlisting}
\documentclass[simple]{elegantbook} %or
\documentclass[mode=simple]{elegantbook}
\end{lstlisting}

In this template, we defined four different theorem class environments

\begin{itemize}
    \item \textit{Theorem Environment}, including title and content, numbering corresponding to chapter. Three types depending on the format:
    \begin{itemize}
        \item \textcolor{main}{\textbf{definition}} environment, the color is  \textcolor{main}{main};
        \item \textcolor{second}{\textbf{theorem, lemma, corollary}} environment, the color is \textcolor{second} {second};
        \item \textcolor{third}{\textbf{proposition}} environment, the color is \textcolor{third}{third}.
    \end{itemize}
    \item \textit{Example Environments}, including \textbf{example, exercise, problem} environment, auto numbering corresponding to chapter.
    \item \textit{Proof Environment}, including \textbf{proof, note} environment containing introductory symbol (\textbf{note} environment) or ending symbol (\textbf{proof} environment).
    \item \textit{Conclusion Environments}, including \textbf{conclusion, assumption, property, remark} and
    \textbf{solution}\footnote{We also define an option \lstinline{result}, which can hide the \lstinline{solution}
    and \lstinline{proof} environments. You can switch between \lstinline{result=answer} and \lstinline{result=noanswer}.} environments, all of which begin with boldfaced words, with format consistent with normal paragraphs.
\end{itemize}

\subsection{Theorem Class Environments}
Since the template uses the \lstinline{tcolorbox} package to customize the theorem class environments, it is slightly different from the normal theorem environments. The usage is as follows:
\begin{lstlisting}
\begin{theorem}{theorem name}{label text}
The content of theorem.
\end{theorem}
\end{lstlisting}

The first parameter \lstinline{<theorem name>} represents the name of the theorem, and the second parameter \lstinline{label} represents the label used in cross-reference with \verb|ref{thm:label}|. Note that cross-references must be prefixed with \lstinline{thm:}.

From version 4.1, you can write your theorem environments as follows:
\begin{lstlisting}
\begin{theorem}[theorem name]\label{thm:label text}
  The content of theorem.
\end{theorem}
% or
\begin{theorem}
  The content of theorem.
\end{theorem}
\end{lstlisting}

Other theorem class environments with the same usage includes:

\begin{table}[htbp]
    \centering
    \caption{Theorem Class Environments}
    \begin{tabular}{llll}
        \toprule
        Environment & Label text & Prefix & Cross-reference             \\
        \midrule
        definition  & label      & def    & \lstinline|\ref{def:label}| \\
        theorem     & label      & thm    & \lstinline|\ref{thm:label}| \\
        lemma       & label      & lem    & \lstinline|\ref{lem:label}| \\
        corrlary    & label      & cor    & \lstinline|\ref{cor:label}| \\
        proposition & label      & pro    & \lstinline|\ref{pro:label}| \\
        \bottomrule
    \end{tabular}%
    \label{tab:theorem-class}%
\end{table}%

\subsection{Other Customized Environments}
The other three math environments can be called directly since there are no additional option for them, e.g. \lstinline{example}:
\begin{lstlisting}
\begin{example}
This is the content of example environment.
\end{example}
\end{lstlisting}

The effect is as follows:

\begin{example}
    This is the content of example environment.
\end{example}

These are all similar environments with slight differences lies in:

\begin{itemize}
    \item Example, exercise, problem environments number within chapter;
    \item Note begins with introductory symbol and proof ends with ending symbol;
    \item Conclusion and other environments are normal paragraph environments with boldfaced introductory words.
\end{itemize}


\section{List Environments}
This template uses \lstinline{tikz} to customize the list environments, with \lstinline{itemize} environment customized to the third depth and \lstinline{enumerate} environment customized to fourth depth. The effect is as follows\\[2ex]
\begin{minipage}[b]{0.49\textwidth}
    \begin{itemize}
        \item first item of nesti;
        \item second item of nesti;
        \begin{itemize}
            \item first item of nestii;
            \item second item of nestii;
            \begin{itemize}
                \item first item of nestiii;
                \item second item of nestiii.
            \end{itemize}
        \end{itemize}
    \end{itemize}
\end{minipage}
\begin{minipage}[b]{0.49\textwidth}
    \begin{enumerate}
        \item first item of nesti;
        \item second item of nesti;
        \begin{enumerate}
            \item first item of nestii;
            \item second item of nestii;
            \begin{enumerate}
                \item first item of nestiii;
                \item second item of nestiii.
            \end{enumerate}
        \end{enumerate}
    \end{enumerate}
\end{minipage}


\section{Fonts}
The \lstinline{math} font option offers:
\begin{enumerate}
    \item \lstinline{math=cm}(default), use \LaTeX{} default math font (recommended).
    \item \lstinline{math=newtx}, use \lstinline{newtxmath} math font (may bring about bugs).
    \item \lstinline{math=mtpro2}, use \lstinline{mtpro2} package to set math font.
\end{enumerate}

\subsection{Symbol Fonts}
Feedback from some 3.08 users claims that error occurs when using our templates with  \lstinline{yhmath}, \lstinline{esvect} and other packages.
\begin{lstlisting}
LaTeX Error:
Too many symbol fonts declared.
\end{lstlisting}

The reason is that the template redefines font for math so that no new math font is allowed to be added. To use \lstinline{yhmath} and/or \lstinline{esvect}, please locate \lstinline{yhmath} or \lstinline{esvect} in \lstinline{elegantbook.cls}, uncomment corresponding related code.

\begin{lstlisting}
%%% use yhmath pkg, uncomment following code
% \let\oldwidering\widering
% \let\widering\undefined
% \RequirePackage{yhmath}
% \let\widering\oldwidering

%%% use esvect pkg, uncomment following code
% \RequirePackage{esvect}
\end{lstlisting}


\section{Bibliography}

This template uses biblatex to generate the bibliography, the default citestyle and bibliography style are both \lstinline{numeric}. Let's take a glance at the citation effect. ~\cite{en1} use data from a major peer-to-peer lending \cite{en3} marketplace in China to study whether female and male investors evaluate loan performance differently \parencite{en2}.

If you want to use biblatex, you must create a file named \lstinline{reference.bib}, add bib items (from Google Scholar, Mendeley, EndNote, and etc.) to \lstinline{reference.bib} file, then cite the bibkey in the \lstinline{tex} file. The biber will automatically generate the bibliography for the reference you cited.


To change the bibliography style, this version introduces \lstinline{citestyle} and \lstinline{bibstyle}, please refer to \href{https://ctan.org/pkg/biblatex}{CTAN:biblatex} for more detail about these options. You can change your bibliography style as
\begin{lstlisting}
\documentclass[citestyle=numeric-comp, bibstyle=authoryear]{elegantbook}
\end{lstlisting}


\section{Preface}

If you want to add a preface before the first chapter with the number of chapter unchanged, please add the preface in the following way:
\begin{lstlisting}
\chapter*{Introduction}
\markboth{Introduction}{Introduction}
The content of introduction.
\end{lstlisting}


\section{Content Option and Depth}
Option for content \lstinline{toc}, you can choose either one column(\lstinline{onecol}) or two columns(\lstinline{twocol}). For two columns:
\begin{lstlisting}
\documentclass[twocol]{elegantbook}
\documentclass[toc=twocol]{elegantbook}
\end{lstlisting}

Default content depth is 1, use to use \lstinline|\setcounter{tocdepth}{2}|.


\section{Introduction Environment}
We create a introduction environment to display the structure of chapter. The basic useage is as follows:
\begin{lstlisting}
\begin{introduction}
  \item Definition of Theorem
  \item Ask for help
  \item Optimization Problem
  \item Property of Cauchy Series
  \item Angle of Corner
\end{introduction}
\end{lstlisting}
And you will get:
\begin{introduction}
    \item Definition of Theorem
    \item Ask for help
    \item Optimization Problem
    \item Property of Cauchy Series
    \item Angle of Corner
\end{introduction}

You can change the title of this environment by modifying the optional argument of this environment:
\begin{lstlisting}
\begin{introduction}[Brief Introduction]
...
\end{introduction}
\end{lstlisting}

%\section{Problem Set}
The environment \lstinline{problemset} is used at the end of each chapter to display corresponding exercises. Just type in the following sentences:
\begin{lstlisting}
\begin{problemset}
  \item exercise 1
  \item exercise 2
  \item exercise 3
\end{problemset}
\end{lstlisting}
And you will get:
\begin{problemset}
    \item exercise 1
    \item exercise 2
    \item exercise 3
    \item math equation test:
    \begin{equation}
        a^2+b^2=c_{2_{i}} (1,2) [1,23]
    \end{equation}
\end{problemset}
\begin{remark}
    If you want to customize the title of \lstinline{problemset}, please change the optional argument like in introduction environment. In this version the \lstinline{problemset} environment automatically appears in the table of contents but not in the header or footer(to be fixed).
\end{remark}

\begin{solution}
    If you want to customize the title of \lstinline{problemset}, please change the optional argument like in introduction environment. In this version the \lstinline{problemset} environment automatically appears in the table of contents but not in the header or footer(to be fixed).
\end{solution}

\chapter{ElegantBook Writing Sample}

\begin{introduction}
    \item Theorem Class Envrionments
    \item Cross Reference
    \item Math Environments
    \item List Environments
    \item Logo and Base
    \item $a^2+b^2=c^2$
\end{introduction}


\section{Writing Sample}

We will define the integral of a measurable function in three steps. First, we define the integral of a nonnegative simple function. Let $E$ be the measurable set in $\mathcal{R}^N$.

\begin{definition}[Left Coset]
    Let $H$ be a subgroup of a group~$G$. A \emph{left coset} of $H$ in $G$ is a subset of $G$ that is of the form $xH$, where $x \in G$ and $xH = \{ xh : h \in H \}$. Similarly a \emph{right coset} of $H$ in $G$ is a subset of $G$ that is of the form $Hx$, where $Hx = \{ hx : h \in H \}$ $\hbar$
\end{definition}

\begin{note}
    Note that a subgroup~$H$ of a group $G$ is itself a left coset of $H$ in $G$.
\end{note}

\begin{theorem}[Lagrange's Theorem]
    \label{thm:lg}
    Let $G$ be a finite group, and let $H$ be a subgroup of $G$. Then the order of $H$ divides the order of $G$.
\end{theorem}

\begin{proposition}[Size of Left Coset]
    Let $H$ be a finite subgroup of a group $G$. Then each left coset of $H$ in $G$ has the same number of elements as $H$.
\end{proposition}

\begin{proof}
    Let $z$ be some element of $xH \cap yH$. Then $z = xa$ for some $a \in H$, and $z = yb$ for some $b \in H$. If $h$ is any element of $H$ then $ah \in H$ and $a^{-1}h \in H$, since $H$ is a subgroup of $G$. But $zh = x(ah)$ and $xh = z(a^{-1}h)$ for all $h \in H$. Therefore $zH \subset xH$ and $xH \subset zH$, and thus $xH = zH$. Similarly $yH = zH$, and thus $xH = yH$, as required.
\end{proof}

\section{Second section}
This second section\index{S!section} may include some special word,
and expand the ones already used\index{keywords!used}.


\begin{table}[htbp]
    \small
    \centering
    \caption{Auto MPG and Price \label{tab:reg}}
    \begin{tabular}{lcc}
        \toprule
        & (1)        & (2)     \\
        \midrule
        mpg      & -238.90*** & -49.51  \\
        & (53.08)    & (86.16) \\
        weight   &            & 1.75*** \\
        &            & (0.641) \\
        constant & 11,253***  & 1,946   \\
        & (1,171)    & (3,597) \\
        obs      & 74         & 74      \\
        $R^2$    & 0.220      & 0.293   \\
        \bottomrule
        \multicolumn{3}{l}{\scriptsize Standard errors in parentheses} \\
        \multicolumn{3}{l}{\scriptsize *** p<0.01, ** p<0.05, * p<0.1} \\
    \end{tabular}%
\end{table}%

%\problemset
\begin{problemset}
    \item Solve the equation $5(- 3x - 2) - (x - 3) = -4(4x + 5) + 13$.
    \item Find the distance between the points $(-4 , -5)$ and $(-1 , -1)$.
    \item Find the slope of the line $5x - 5y = 7$.
\end{problemset}
