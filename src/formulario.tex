\chapter{Formulario}\label{ch:formulario}
En este capítulo vamos a recoger todas las ecuaciones, fórmulas, resultados que hemos usado de forma ?

\section{Análisis matemático}\label{sec:analisis-matematico}

\subsection{Polinomio de Taylor}\label{subsec:polinomio-de-taylor}
Sea $f$ una función real definido en un intervalo abierto y $x_0$ un número del dominio, existe $\epsilon>0\in\R$ tal que $\forall\ x\in(x_0-\epsilon\coma x_0+\epsilon)$ se tiene que
\begin{equation}
    \label{eq:polinomio-taylor}
    f(x)=\sum_{n=0}^{\infty} \frac{f^{(n)}(x_0)}{n !}(x-x_0)^n
\end{equation}

Sea $f$ una aplicación de $\R^2$ en $\R^2$, definido en un intervalo abierto y $(x_0, y_0)$ un elemento del dominio, existe $\epsilon > 0 \in\R$ tal que $\forall\ (x, y)\in(x_0-\epsilon\coma x_0+\epsilon)\times(y_0-\epsilon, y_0+\epsilon)$ se tiene que
\begin{equation}
    \label{eq:polinomio-taylor-dos-variables}
    f(x,y)=\sum_{n=0}^{\infty} \sum_{m=0}^{\infty} \frac{\partial^{n+m}f(x_0, y_0)}{\partial^n x\partial^m y}\frac{(x-x_0)^n(y-y_0)^m}{n!m!}
\end{equation}

\section{Ecuaciones diferenciables}\label{sec:ecuaciones-diferenciables}
\subsection{Polinomio de segundo grado}\label{subsec:polinomio-de-segundo-grado}
La solución a la ecuación polinómica de segundo grado $ay^{''}+by^{'}+c=0$, se basa en calcular las raices $r_1\coma r_2$ del polinomio
$ax^2+bx+c$
\begin{itemize}
    \item Si $r_1\neq r_2\in\R\so y(x)=c_{1}e^{r_{1}x}+c_{2}e^{r_{2}x}$.
    \item Si $r_1= r_2\in\R\so y(x)=(c_{1}+c_{2}x)e^{r_{1}x}$.
    \item Si $r_1= \overline{r_2}=\alpha+i\beta\in\C\so y(x)=(c_{1}\cos(\beta x)+c_{2}\sin(\beta x))e^{\alpha x}$.
\end{itemize}

\section{Integrales}\label{sec:integrales}
\subsection{Integral de Gauss}\label{subsec:integral-de-gauss}
\subsubsection*{Integral de Gauss simple}
\begin{equation}
    \label{eq:integral-gauss}
    \int_{-\infty}^{+\infty} e^{-x^{2}}dx =\sqrt{\pi}
\end{equation}

\subsubsection*{Integral de Gauss generalizada}
\begin{equation}
    \label{eq:integral-gauss-generalizada}
    \int_{-\infty}^{+\infty} ae^{-(x+b)^{2}/c^2}dx =a\abs{c}\sqrt{\pi}
\end{equation}

\subsubsection*{Integral de Gauss con potencias pares}
\begin{equation}
    \label{eq:integral-gauss-generalizada-potencias-pares}
    \int_{-\infty}^{+\infty} x^{2n}e^{-\frac{1}{2}ax^{2}}dx =\left(\frac{2\pi}{a}\right)^{\frac{1}{2}}\frac{1}{a^n}(2n-1)!!
\end{equation}
