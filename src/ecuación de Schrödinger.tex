\chapter{Ecuación de Schrödinger}

\section{Función de onda}
A la onda asociada a una partícula se le llama \define{función de onda}{funcion_de_onda} y se le suele denotar por $\Psi$.

La \define{ecuación de Schrödinger}{ecuacion_de_schrodinger}, describe la evolución temporal de la función de onda de una partícula masiva y no relativista.
\begin{postulate}[Ecuación de Schrödinger]
    \begin{equation}
    \label{eq:ecuacion_schrodinger}
    \frac{\partial\Psi(\vec{r},t)}{\partial t}=\frac{i\hbar}{2m}\nabla\Psi(\vec{r},t)-\frac{i}{\hbar}V(\vec{r},t)\Psi(\vec{r},t)
    \end{equation}
\end{postulate}

\section{Propiedades de las soluciones a la ecuación de Schrödinger}\label{sec:propiedades-de-las-soluciones-a-la-ecuación-de-schrödinger}

\subsection{Linealidad}\label{subsec:linealidad}
Es fácil comprobar que si $\Psi_1$ y $\Psi_2$ son funciones de onda para un potencial $V(x,t)$ y $a\in\C$, entonces $\Psi_1+\Psi_2$ y $a\Psi$ son ambos funciones de onda para el mismo potencial $V(x,t)$.
En cuanto a la conjugada tenemos que $\Psi^*_1$ es el opuesto a la solución de la ecuación de onda del conjudado del potencial, es decir

\begin{equation}
    \label{eq:ecuacion_schrodinger_conjugada}
    \frac{\partial\Psi^*(\vec{r},t)}{\partial t}=\frac{-i\hbar}{2m}\nabla\Psi^*(\vec{r},t)+\frac{i}{\hbar}V^*(\vec{r},t)\Psi^*(\vec{r},t)
\end{equation}

\subsubsection{Solución temporal}
Si la función de onda sólo depende del tiempo, la ecuación~\eqref{eq:ecuacion_schrodinger} queda como $\Psi^\prime(t) = -i/\hbar V(t)\Psi(t)$ cuya solución es $\Psi(t)=e^{-it(W(t)+W)/\hbar}$ donde $W^\prime (t)=V(t)$ y $W\in\C$.

\subsubsection{Solución espacial}
Si la función de onda sólo depende del espacio, la ecuación~\eqref{eq:ecuacion_schrodinger} queda como $0=-i\hbar/(2m)\nabla\Psi(\vec{r})+V(\vec{r})\Psi(\vec{r})$, es decir $\nabla\Psi(\vec{r})=-i2m/\hbar V(\vec{r})\Psi(\vec{r})$.

\subsection{Estados estacionarios}\label{subsec:estados-estacionarios}
Vamos a estudiar un caso muy especial, donde el potencial no depende del tiempo y donde la solución es el producto de una solución espacial y otra temporal, simplificaremos las ecuaciones tomando una sola variable, es decir $\Psi(x,t)=f(x)g(t)$, la ecuación~\eqref{eq:ecuacion_schrodinger} queda como $f(x)g'(t)=i\hbar/(2m)f''(x)g(t)-i/\hbar V(x)f(x)g(t)$.


\textbf{Caso 1.}

Para algún momento $t_0$ la función $g$ se anula, es decir, $f(x)g'(t_0)=0$ para todo punto del espacio, así que $g'(t_0)=0$.

\textbf{Caso 2.}

Para algún punto del espacio $x_0$ la función $f$ se anula, es decir, $0=f''(x_0)g(t)$ en todo momento, así que $f''(x_0)=0$.

\textbf{Caso 3.}

Nunca se anulan las funciones, podemos entonces dividir en todo momento la ecuación anterior por $f(x)g(t)$, quedando la ecuación $g'(t)/g(t)=i\hbar/(2m)f^{\prime\prime}(x)/f(x)-i/\hbar V(x)$.
Teniendo que una expresión temporal es igual a una expresión espacial para todo tiempo y todo espacio, esto es posible si ambas expresiones son constantes en tiempo y espacio, y deben cumplirse a la vez que para algún $C\in\C$
\begin{gather*}
    \frac{g'(t)}{g(t)}=C\Rightarrow g(t)=e^{Ct}
    \\
    \frac{i\hbar}{2m}\frac{f''(x)}{f(x)}-\frac{i}{\hbar}V(x)=C
\end{gather*}

\begin{resultado}
    Si el potencial $V$ es independiente del tiempo y $C\in\C$ cumple que $\frac{g'(t)}{g(t)}=C$, entonces $C$ es imaginario.
\end{resultado}
\begin{proof}
    Si calculamos la norma al cuadrado de $\Psi$ tenemos que
    \begin{equation*}
        \nor{\Psi(x,t)}^2=\Psi^*(x,t)\Psi(x,t)=f^*(x)g^*(t)f(x)g(t)=\nor{f(x)}^2\nor{g(t)}^2
    \end{equation*}
    Usando la ecuación~\eqref{eq:ecuacion_probabilidad_onda} tenemos que
    \begin{equation*}
        1=\int_{-\infty}^{+\infty}\nor{\Psi(x,t)}^2 dx=\nor{g(t)}^2\int_{-\infty}^{+\infty}\nor{f(x)}^2 dx
    \end{equation*}
    Para que la ecuación anterior sea cierta, la norma de $g$ debe ser una constante, es decir, $g^*(t)g(t)=e^{C^*t}e^{Ct}=e^{(C^*+C)t}$ debe ser constante en el tiempo, por tanto se tiene como única opción que $C^*+C=0$, es decir, que $C$ sea un número imaginario.
\end{proof}

\begin{resultado}
    Si el potencial $V$ es independiente del tiempo y $\Psi(x,t)=f(x)g(t)$ se cumple que $\nor{\Psi(x,t)}^2=\nor{f(x)}^2$.
\end{resultado}
\begin{proof}
    Como hemos visto en el resultado anterior, $\nor{g(t)}^2=1$, y por tanto
    \begin{equation}
        \label{eq:probabilidad_atemporal}
        \nor{\Psi(x,t)}^2=\nor{f(x)}^2\nor{g(t)}^2=\nor{f(x)}^2
    \end{equation}
\end{proof}
Así que $C=iD$ con $D\in\R$, las soluciones al estado estacionario queda como
\begin{gather*}
    \frac{g'(t)}{g(t)}=iD
    \\
    \frac{\hbar}{2m}\frac{f''(x)}{f(x)}-\frac{1}{\hbar}V(x)=D
\end{gather*}

\subsubsection{Hamiltoniano}
Si consideramos el conjunto de la energía cinética más la energía potencial, obtenemos lo que llamamos el \define{Hamiltoniano} y tiene la expresión
\begin{equation}
    \label{eq:halmitoniano}
    H(x,p)=\frac{p^2}{2m}+V(x)
\end{equation}
Llamamos \define{operador Hamiltoniano} al operador definido por
\begin{equation}
    \label{eq:operador_hamiltoniano}
    \widehat{H}=-\frac{\hbar^2}{2m}\frac{\partial^2}{\partial x^2}+V(x)
\end{equation}
Usando el operador Hamiltoniano en la ecuación~\eqref{eq:ecuacion_schrodinger_tempoespacial} podemos expresarlo así
\begin{equation*}
    \widehat{H}f(x)=-i\hbar Df(x)
\end{equation*}
Para simplificar la expresión llamamos $E=-\hbar D$ y por tanto las soluciones al estado estacionario queda como
\begin{gather*}
    \frac{g'(t)}{g(t)}=-iE/\hbar
    \\
    \frac{\hbar}{2m}\frac{f''(x)}{f(x)}-\frac{1}{\hbar}V(x)=-E/\hbar\Rightarrow \frac{\hbar^2}{2m}\frac{f''(x)}{f(x)}-V(x)=-E
\end{gather*}

\begin{definicion}
    Llamamos \define{Ecuación de Schrödinger independiente del tiempo} a la ecuación solución de la parte espacial del estado estacionario
    \begin{equation}
        \label{eq:ecuacion_schrodinger_tempoespacial}
        -\frac{\hbar^2}{2m}f''(x)+V(x)f(x)= Ef(x)
    \end{equation}
\end{definicion}

Resolviendo la ecuación diferencial temporal y usando el operador de Halmiton podemos dejar las soluciones de estados estacionarios como
\begin{equation}
    \label{eq:solucion_estado_estacionario}
    \Psi(x,t)=f(x)g(t)\text{ donde }
    \begin{cases}
        g(t)=e^{-iEt/\hbar}\\
        \widehat{H}f(x)=Ef(x)\\
        E\in\R
    \end{cases}
\end{equation}

\subsection{Observaciones a las soluciones del estado estacionario}\label{subsec:observaciones-a-las-soluciones-del-estado-estacionario}

\subsubsection{Solución general}
La solución general del estado estacionario se puede expresar como
\begin{equation}
    \label{eq:solucion_general_estado_estacionario}
    \Psi(x,t)=\sum_{k\in\N}c_k f_k(x)e^{-iE_k t/\hbar}
\end{equation}
Donde $\forall k\in\N$ $E_k\in\R$ y $\widehat{H}f_k(x)=E_k f_k(x)$

\subsubsection{Desviación de H}
$\avg{H} = \int_{-\infty}^{+\infty}\Psi^* \widehat{H}\Psi dx = \int_{-\infty}^{+\infty}\Psi^* E\Psi dx=E\int_{-\infty}^{+\infty}\Psi^* \Psi dx = E$

$\avg{H^2} = \int_{-\infty}^{+\infty}\Psi^* \widehat{H}^2\Psi dx = \int_{-\infty}^{+\infty}\Psi^* E^2\Psi dx=E^2\int_{-\infty}^{+\infty}\Psi^* \Psi dx = E^2$

Por tanto $\sigma_H=\avg{H^2}-\avg{H}^2=E^2-E^2=0$
