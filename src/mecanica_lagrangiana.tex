\chapter{Formulación mecánica Lagrangiana}


\section{El lagrangiano del sistema}

El lagrangiano es una aplicación escalar definida sobre un cierto espacio de posibles estados del sistema, donde dicho sistema viene determinado en cada momento por la posición y velocidad de cada uno de sus componentes.

El espacio de estados es una variedad diferenciable construida como el fibrado tangente $T\mathcal{M}$ de una variedad $\mathcal{M}$ y el lagrangiano es una función de la forma $\maps{L}{T\mathcal{M}\times\R}{\R}$.

Para la física clásica, el lagrangiano de un sistema, se forma mediante la energía cinética ($T$) y la energía potencial ($V$) de tal manera que:
\begin{postulate}[Lagrangiano clásico en base a la energía]
    \begin{equation}
        \label{eq:lagrangiano_clasico}
        L(x,\dot{x},t)=T-V=\frac{1}{2}m\dot{x}(t)^2-V(x,t)
    \end{equation}
\end{postulate}


\section{Principio de mínima acción}

El principio de mínima acción o principio de Hamilton, es un postulado básico de la mecánica clásica y la mecánica relativista para describir la evolución a lo largo del tiempo del estado de movimiento de una partícula como de un campo físico.

Históricamente, el principio de mínima acción postulaba que, la evolución temporal de todo sistema físico se daba de tal manera que una cantidad denominada <<acción>> tendía a ser la mínima posible.

Si $\maps{\alpha}{I=[t_1, t_2]}{\R^n}$ es la trayectoria de una partícula y $\maps{L}{\R^n\times\R^n\times\R}{\R}$ describe el sistema en base a su posición y su velocidad, definimos la acción $S$ como:
\begin{postulate}[Acción del sistema sobre una trayectoria]
    \begin{equation}
        \label{eq:accion}
        S(\alpha,I)=\int_{I}L(\alpha(t), \dot{\alpha}(t), t)\ dt
    \end{equation}
\end{postulate}

El \textbf{principio de mínima acción} o \textbf{principio de Hamilton}, dice que la trayectoria que una partícula toma, es aquella para la cual, la acción sea mínima.
Podemos deducir, que la minimicidad de la acción es equivalente a que se cumpla la ecuación del movimiento de Euler-Lagrange:
\begin{postulate}[Ecuación de Euler-Lagrange del movimiento]
    \begin{equation}
        \label{eq:euler_lagrange}
        \frac{d}{dt}\left(\frac{\partial L}{\partial \dot{\alpha}}\right)=\frac{\partial L}{\partial \alpha}
    \end{equation}
\end{postulate}


\section{Energía}
La energía de un sistema es la suma de la energía cinética y la energía potencial $E=K+V$, si ponemos esta expresión en terminos del lagrangiano, tenemos que $E=T+V=T+(T-L)=2T-L$.

Es fácil comprobar que $\frac{\partial L}{\partial\dot{\alpha}}\dot{\alpha}=m\dot{\alpha}^2=2T$, obteniendo las siguientes expresiones de la energía:
\begin{equation}
    \label{eq:energia_lagrangiana}
    E(\alpha, \dot{\alpha})=L(\alpha, \dot{\alpha})+m\dot{\alpha}^2=\frac{\partial L}{\partial\dot{\alpha}}\dot{\alpha}-L
\end{equation}


\section{Conservación de la energía}
Es lógico pensar que las leyes de la física no varian con el tiempo, esto nos obliga a exigir que el lagrangiano de un sistema sea el mismo con respecto al tiempo, así pues, debemos tener que $\dot{L}=0$.
Aplicando la regla de la cadena tenemos:
\begin{equation*}
    \dot{L}=\frac{dL(\alpha, \dot{\alpha})}{dt}=\frac{\partial L}{\partial \alpha}\frac{d\alpha}{dt}+\frac{\partial L}{\partial \dot{\alpha}}\frac{d\dot{\alpha}}{dt}\by{\ref{eq:euler_lagrange}}\frac{d}{dt}\left(\frac{\partial L}{\partial \dot{\alpha}}\right)\dot{\alpha}+\frac{\partial L}{\partial \dot{\alpha}}\frac{d\dot{\alpha}}{dt}=\frac{d}{dt}\left(\frac{\partial L}{\partial \dot{\alpha}}\dot{\alpha}\right)\by{\ref{eq:energia_lagrangiana}}\frac{d}{dt}\left(E+L\right)=\dot{E}+\dot{L}
\end{equation*}
La igualdad anterior nos lleva a la expresión $\dot{E}=0$, es decir, la energía no varía con el tiempo.


\section{Las leyes de Newton a través del lagrangiano}
La tercera ley de Newton, dice que la fuerza es la masa por la aceleración, como la fuerza viene regida por el potencial $-\frac{\partial V}{\partial\alpha}$ y la aceleración es la variación de la velocidad, la tercera ley de Newton queda en:
\begin{equation*}
    -\frac{\partial V}{\partial\alpha} = m\frac{d\dot{\alpha}}{dt}
\end{equation*}
Ahora tenemos las siguientes igualdades:
\begin{subequations}
    \begin{equation*}
        L=\frac{1}{2}m\dot{\alpha}^2-V(\alpha).
    \end{equation*}
    \begin{equation*}
        \frac{\partial L}{\partial \alpha}=-\frac{\partial V}{\partial \alpha}
    \end{equation*}
    \begin{equation*}
        \frac{\partial L}{\partial\dot{\alpha}}=m\dot{\alpha}
    \end{equation*}
\end{subequations}


\section{Acciones equivalentes}
\begin{definition}
    Diremos que dos trayectorias $\maps{\alpha,\beta}{[t_1, t_2]}{\R^n}$ son \define{equivalentes}{trayectorias equivalentes} si sus acciones tienen el mismo valor.
\end{definition}

Dos trayectorias $\alpha, \beta$ son equivalentes, si la acción sobre su diferencia es cero, por lo tanto $\int_{t_1}^{t_2} L(\alpha, \dot{\alpha}, t)-L(\beta, \dot{\beta}, t)=0$, es decir, existe una función $\maps{f}{\R^n\times\R^n\R}{\R}$ que es la primitiva de $L(\alpha, \dot{\alpha})-L(\beta, \dot{\beta})$ con respecto al tiempo para cual $f(\alpha(t_2))-f(\alpha(t_1))=0$ y $L(\alpha, \dot{\alpha})=L(\beta, \dot{\beta})+\frac{df}{dt}$
