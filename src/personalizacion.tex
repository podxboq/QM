\providecommand{\tq}{\mid}
\providecommand{\N}{\mathbb{N}}
\providecommand{\Z}{\mathbb{Z}}
\providecommand{\Q}{\mathbb{Q}}
\providecommand{\R}{\mathbb{R}}
\providecommand{\C}{\mathbb{C}}
\providecommand{\H}{\mathcal{H}}
\providecommand{\Im}[1]{Im(#1)}
\providecommand{\Re}[1]{Re(#1)}
\providecommand{\conjugate}[1]{\bar{#1}}
\providecommand{\pescalar}[2]{\langle #1,#2 \rangle}
\providecommand{\braket}[2]{\left\langle#1\mid#2\right\rangle}
\providecommand{\bra}[1]{\left\langle#1\right\rvert}
\providecommand{\ket}[1]{\left\lvert#1\right\rangle}
\providecommand{\so}{\Rightarrow}
\providecommand*{\circled}[1]{\tikz[baseline=(char.base)]{\node[shape=circle,draw,inner sep=2pt] (char) {#1};}}
\providecommand{\by}[1]{\overset{\fbox{\tiny #1}}{=}}
\providecommand{\maps}[3]{#1:#2\longrightarrow #3}
\providecommand{\coma}{,\thinspace}
\providecommand{\pari}[2]{(#1,\thinspace #2)}
\providecommand{\cartalocal}{(\mathcal{M}^n,\thinspace p,\thinspace U,\thinspace \varphi)\textmd{ una carta local}}
\providecommand{\doscartaslocales}{(\mathcal{M}^n\coma p\coma U\coma\varphi)\textmd{ y } (\mathcal{N}^m\coma q\coma
V\coma\psi)\textmd{ dos cartas locales}}
\providecommand{\indexdots}[3]{#1=#2,\ldots,#3}
\providecommand{\define}[2]{\textbf{#1}\label{def:#2}}
\providecommand{\avg}[1]{\left\langle#1\right\rangle}
\providecommand{\abs}[1]{\lvert#1\rvert}
\providecommand{\nor}[1]{\lVert#1\rVert}
\providecommand{\operatoravg}[3]{\left\langle#1|#2|#3\right\rangle}
\providecommand{\cinfinity}[1]{\mathscr{C}^\infty(#1)}
\providecommand{\mapsdef}[5]{ #1:\ #2 & \longrightarrow #3 \\ #4 & \longmapsto #5}
\providecommand{\glossarydef}[3]{\newglossaryentry{#1}{name={#2},description={#3}}\gls{#1}}
\providecommand{\lie}[2]{[#1\coma #2]}
\providecommand{\metrica}{\mathsf{g}}
\providecommand{\vec}[1]{\overrightarrow{#1}}
\newcommand{\set}[1]{\left\{#1\right\}}
\newcommand{\where}{\mathrel{}\middle|\mathrel{}}
\renewcommand*{\hbar}{{\mkern-1mu\mathchar'26\mkern-8mu\mathrm{h}}}
\renewcommand{\c}{\mathrm{c}}
\newcommand{\h}{\mathrm{h}}
% Box color
\newcommand{\colorequation}[3]{\tcboxmath[on line,
    fonttitle=\small\sffamily,
    colbacktitle=#2!10!white,coltitle=#2!50!black,
    title=\small{#1},
    arc=0pt,outer arc=0pt,
    colback=#2!10!white,colframe=#2!50!black,
    boxsep=1pt,left=1pt,right=1pt,top=1pt,bottom=1pt,
    boxrule=0pt]{#3}}

% From mandi package
\newcommand{\usk}{\cdot}
\newcommand{\kilogram}{\mathrm{kg}}
\newcommand{\meter}{\mathrm{m}}
\newcommand{\second}{\mathrm{s}}
\newcommand{\inverse}{^{-1}}           % postfix -1
\newcommand{\tothetwo}{^2}             % postfix  2
\newcommand{\tothethree}{^3}           % postfix  3
\newcommand{\tothefour}{^4}            % postfix  4
\newcommand{\tento}[1]{10^{#1}}
\newcommand{\xtento}[1]{\times\tento{#1}}
\providecommand{\speedoflightsymbol}{\c}
\providecommand{\speedoflightvalue}{2.99792458\times10^{8}}
\providecommand{\speedoflightounits}{\meter\usk\second\inverse}
\providecommand{\plancksymbol}{\h}
\providecommand{\planckvalue}{6.62607015\times10^{-34}}
\providecommand{\planckunits}{\kilogram\usk\meter\tothetwo\usk\second\inverse}
\providecommand{\planckbarsymbol}{\hbar}
\providecommand{\planckbarvalue}{1.054571817\times10^{-34}}
\providecommand{\planckbarunits}{\kilogram\usk\meter\tothetwo\usk\second\inverse}
\newcommand\ct[1]{\text{\rmfamily\upshape #1}}